\documentclass[article]{dennou777}

\ltjsetparameter{jacharrange={-2}}%ギリシア文字、キリル文字を AL Char に。
\usepackage[no-math,match,deluxe,fontspec,jfm_yoko=jlreq,jfm_tate=jlreqv]{luatexja-preset}

\usepackage[T1]{fontenc}
\usepackage{yhmath,amsmath,mathtools,amssymb,mathrsfs,rsfso,mleftright}
\usepackage[math]{kurier}
\usepackage[euler-digits]{eulervm}
\usepackage[scaled]{beramono}
%\usepackage{classico,newpxtext}
\DeclareMathAlphabet{\mathtt}{T1}{fvm}{m}{n}
\DeclareMathAlphabet{\mathsf}{T1}{uop}{m}{n}

\usepackage[unicode,colorlinks]{hyperref}
\hypersetup{linkcolor=blue,urlcolor=teal,citecolor=olive}

\usepackage{pxrubrica}
\usepackage{autobreak}
\usepackage{tcolorbox}
\usepackage{epmaketitle} %https://github.com/matryo-sika/epmaketitle
\renewcommand{\maketitle}{\epmaketitle}

\setmainfont{Exo 2}
\setsansfont{Exo 2}
\setmainjfont{FOT-MatissePro-DB}[
	AltFont={
		{
			Range={
				"4E00-"9FFF, % CJK 統合漢字
				"3400-"4DFF, % CJK 統合漢字拡張 A
				"20000-"2EBE0, % CJK 統合漢字拡張 B-F
				"2460-"24FF, % 囲み英数字
				"3200-"32FF, % 囲み CJK 文字・月
				"1F100-"1F2FF % 囲み英数字補助、漢字補助
			},
			Font=FOT-RodinNTLGPro-DB,
		},
	},
	BoldFeatures={
		Font=FOT-MatissePro-EB,
		AltFont={
			 {
				Range={
					"4E00-"9FFF, % CJK 統合漢字
					"3400-"4DFF, % CJK 統合漢字拡張 A
					"20000-"2EBE0, % CJK 統合漢字拡張 B-F
					"2460-"24FF, % 囲み英数字
					"3200-"32FF, % 囲み CJK 文字・月
					"1F100-"1F2FF % 囲み英数字補助、漢字補助
				},
				Font=FOT-RodinNTLGPro-EB,
			},
		},
	},
	YokoFeatures={JFM=jlreq},   % jlreqのJFMを維持する
	TateFeatures={JFM=jlreqv},  % https://qiita.com/zr_tex8r/items/91ae1dcc9c3afce7fa8c
]
\setsansjfont{FOT-RodinNTLGPro-DB}[
	AltFont={
		{
			Range={
				"4E00-"9FFF, % CJK 統合漢字
				"3400-"4DFF, % CJK 統合漢字拡張 A
				"20000-"2EBE0, % CJK 統合漢字拡張 B-F
				"2460-"24FF, % 囲み英数字
				"3200-"32FF, % 囲み CJK 文字・月
				"1F100-"1F2FF % 囲み英数字補助、漢字補助
			},
			Font=FOT-RodinNTLGPro-DB,
		},
	},
	BoldFeatures={
		Font=FOT-MatissePro-EB,
		AltFont={
			 {
				Range={
					"4E00-"9FFF, % CJK 統合漢字
					"3400-"4DFF, % CJK 統合漢字拡張 A
					"20000-"2EBE0, % CJK 統合漢字拡張 B-F
					"2460-"24FF, % 囲み英数字
					"3200-"32FF, % 囲み CJK 文字・月
					"1F100-"1F2FF % 囲み英数字補助、漢字補助
				},
				Font=FOT-RodinNTLGPro-EB,
			},
		},
	},
	YokoFeatures={JFM=jlreq},   % jlreqのJFMを維持する
	TateFeatures={JFM=jlreqv},  % https://qiita.com/zr_tex8r/items/91ae1dcc9c3afce7fa8c
]\setmonofont[
	Ligatures=TeX,
	Scale=0.89,
]{DejaVu Sans Mono}
\setmonojfont[
	Ligatures=TeX,
	Scale=0.89,
]{NotoSansMonoCJKjp-Regular}

\allowdisplaybreaks[4]
\ltjenableadjust[lineend=extended,priority=true,profile=true,linestep=true]

%%%%%%%%%%%%自作マクロ

%%matrix
\newcommand{\Mtx}[1]{\begin{matrix} #1 \end{matrix}}
\newcommand{\pMtx}[1]{\begin{pmatrix} #1 \end{pmatrix}}
\newcommand{\bMtx}[1]{\begin{bmatrix} #1 \end{bmatrix}}
\newcommand{\BMtx}[1]{\begin{Bmatrix} #1 \end{Bmatrix}}
\newcommand{\vMtx}[1]{\begin{vmatrix} #1 \end{vmatrix}}
\newcommand{\VMtx}[1]{\begin{Vmatrix} #1 \end{Vmatrix}}

\newcommand{\hmvec}{\mathbold}
\newcommand{\eqdef}{\stackrel{\mathrm{def}}{=}}
\newcommand{\centeralign}[1]{\rule{0pt}{0pt}\hfill#1\hfill\rule{0pt}{0pt}}
\newcommand{\Unit}[1]{\,\mathrm{#1}}

\newcommand{\hmemph}[1]{\textbf{#1}}

\Dtitle[大気放射の基礎]{大気放射の基礎\\--Liou著 藤枝・深堀訳 (2014) の講読--}
\Dauthor{人見祥磨}
\studentid{02160611}
\faculty{北海道大学理学部地球惑星科学科\\惑星宇宙グループ}
\advisor{石渡正樹}

\begin{document}

\begin{titlepage}
	\maketitle
\end{titlepage}

\begin{abstract}
	惑星大気の熱輸送についてシュミレーションをするために、大気の温度分布を
	知りたい。そのために必要な知識を整理した。また、シュミレーションのために
	必要な計算についての理論についても整理した。
\end{abstract}

\tableofcontents
\pagebreak

\section{動機}

ゼミでハドレー循環について勉強するにつれて、惑星大気の熱輸送について興味を持った。
そこで、惑星大気について、自転周期や地軸の傾きによって、どのように熱輸送が変化する
シミュレーションしたいと思った。そのためには大気の温度分布を計算する必要がある。
その計算をするために、放射に関して、観測なども含めて基礎事項を解説している Liou
の教科書~\cite{liou} を講読し、基礎的な事項について確認することとした。

\section{基本的な放射量}

放射が進行することで、放射は相互作用により増減する。これを記述する
\hmemph{放射伝達方程式}を導く。そのために、放射量を表すための物理量を定義する。

\includegraphics[width=\linewidth]{eq.jpg}

\hmemph{単色の放射強度} $I_\lambda$ を導入する。
面積 $dA$ を横切り、$dA$ の法線からなす角 $\theta$ の方向にある微小立体角 $d\Omega$
から入射する、ある波長区間 $\lambda$ 〜 $\lambda+d\lambda$ における、微小時間 $dt$
の間の微小放射エネルギー量 $dE_\lambda$ を考える。このとき、$dE_\lambda$ は単色の
放射強度$I_\lambda$ に関する式として、
\begin{equation}
	dE_\lambda=I_\lambda\cos[\theta]\,dA\,d\Omega\,d\lambda\,dt
\end{equation}
で表されるとする。すなわち、単色の放射強度は以下の式で定義される。
\begin{equation}
	I_\lambda=\frac{dE_\lambda}{\cos[\theta]\,dA\,d\Omega\,d\lambda\,dt}。
\end{equation}
式より明らかなように、単色の放射強度には、放射が流れる方向が含まれている。

半球の全立体角にわたって積分された $I_\lambda$ の法線成分は、
\hmemph{単色の放射フラックス密度} $F_\lambda$ と呼ばれる。
\begin{equation}
	F_\lambda=\int_\Omega I_\lambda\cos[\theta]\,d\Omega
\end{equation}
特に、等方性放射、すなわち放射強度が方向に依存しない場合は、単色の放射フラックス密度は
\begin{equation}
	F_\lambda=\pi I_\lambda
\end{equation}
となる。

\hmemph{全放射フラックス密度} $F$ は単色の放射フラックス密度を、波長全体で積分したものである。
\begin{equation}
	F=\int^\infty_0 F_\lambda\,d\lambda
\end{equation}
全放射フラックス密度を、面全体で積分したものは\hmemph{全放射フラックス} $f$ と呼ばれる。
\begin{equation}
	f=\int_AF\,dA
\end{equation}

\section{散乱と吸収の概念}

放射は、散乱や吸収という過程を経て、強度が変化する。
このことは、放射伝達方程式に記述されるが、実際にどのような現象が発生しているか確認する。

入射波と衝突した粒子が、入射した電磁波のエネルギーを、あらゆる方向に
再放射する過程は\hmemph{散乱}と呼ばれる。散乱はすべての波長で発生し、粒子の大きさが影響する。
粒子の大きさが散乱に及ぼす影響は、サイズパラメーター $x=2\pi a/\lambda$ ($a$ は粒径)に
よって推定することができる。また、散乱は
粒子が少ないとき、個々の粒子が全く同じように散乱する\hmemph{独立散乱}と、
粒子が多量にあるとき、散乱が繰り返される\hmemph{多重散乱}に分けられる。

\includegraphics[width=\linewidth]{mscatter.jpg}

\centeralign{\includegraphics[width=0.8\textwidth]{scatter.jpg}}

\section{黒体放射の法則}

\hmemph{黒体}とは吸収のない理想的な物体である。
放射を考える上で、最も基本的な物体として黒体を考える。
これより、黒体の放射を支配する 4 つの法則について述べる。

\subsection{プランクの法則}
プランク (Planck, 1901) は、原子が黒体表面に小さな電磁振動子のような振る舞いをしていると
仮定し、振動子が持つエネルギーが
\begin{equation}
	E=nh\tilde\nu
\end{equation}
に量子化されると仮定した。ここで、$n\in\mathbb{Z}$ は量子数、$h$ はプランク定数、
$\tilde\nu$ は振動子の周波数である。

放出されたエネルギーを求めるためには、マクスウェル・ボルツマン統計に従った、起こりうる
全ての状態の周波数 $\tilde\nu$ から振動子の総数を知る必要がある。以上の仮定をもとに、
振動子あたりの平均放射エネルギーで正規化すると、射出された単色の放射強度を表す式として、
周波数と絶対温度の関数であるプランク関数が次のように求まる。
\begin{equation}
	B_{\tilde{\nu}}[T]=\frac{2h\tilde{\nu}^3}{c^2(\exp[h\tilde{\nu}/k_\mathrm{B}T]-1)}
\end{equation}
ここで、$h$ はプランク定数、 $k_\mathrm{B}$ はボルツマン定数、$c$ は光速、$T$ は絶対温度
である。プランク定数とボルツマン定数は実験によって決定され、
$h=6.626\times10^{-34}\Unit{J\,s}$、$K_\mathrm{B}=1.3806\times10^{-23}\Unit{J\,K^{-1}}$
である。さらに、波長 $\lambda=c/\tilde\nu$ に関する式に書き直すと、
\begin{equation}
	B_\lambda[T]=\frac{2hc^2}{\lambda^5(\exp[hc/k_\mathrm{B}\lambda T]-1}=
	\frac{C_1\lambda^{-5}}{\pi(\exp[C_2/\lambda T]-1)}
\end{equation}
を得る。ここで、$C_1=2\pi hc^2$、$C_2=hc/k_\mathrm{B}$ である。

\includegraphics[width=\textwidth]{planck.jpg}

\subsection{ステファン・ボルツマンの法則}
黒体の全放射強度を求めるためには、プランク関数を波長全体で積分すればよい。ここで、
$b=2\pi^4k_\mathrm{B}^4/(15c^2h^3)$ を導入すると、
\begin{equation}
	B[T]=\int^\infty_0 B_\lambda[T]\,d\lambda=bT^4
\end{equation}
と求められる。さらに、等方な黒体によって放射される放射フラックス密度を求めると、
\begin{equation}
	F=\pi B[T]=\sigma T^4
\end{equation}
となる。ここで、$\sigma=5.67\times10^{-8}\Unit{J\,m^{-2}\,s^{-1}\,k_\mathrm{B}^{-4}}$は
ステファン・ボルツマン定数と呼ばれる。

この式は、黒体によって放出される放射フラックス密度が、絶対温度の $4$ 乗に比例することを
示していて、\hmemph{ステファン・ボルツマンの法則}と呼ばれる。これは、広い波長域での
赤外放射伝達の解析をする際に基礎となる法則である。

\subsection{ウィーンの変位則}
黒体放射の最大放射強度の波長が、温度に反比例することを、\hmemph{ウィーンの変位則}と呼ぶ。

黒体放射の最大放射強度の波長を求めるには、プランク関数を波長で偏微分し、その偏微分係数が
$0$ になる波長を求めれば良い。
\begin{equation}
	\frac{\partial B_\lambda[T]}{\partial\lambda}=0
\end{equation}
これを解くと
\begin{equation}
	\lambda_m=\frac{a}{T}
\end{equation}
を得る。ここで、$a=2.897\times10^{-3}\Unit{m\,K}$ はウィーンの変位定数と呼ばれている。

\subsection{キルヒホッフの法則}
\hmemph{キルヒホッフの法則}は、射出と吸収が等しくなるという法則である。
キルヒホッフの法則が成り立つためには、以下の条件を満たしていなければならない。
すなわち、温度が一様であり、等方性放射が達成される、熱力学的平衡の条件が必要である。

この条件は、中間圏よりも高度が低い局所的に限られた空間で、
分子の衝突によってエネルギー遷移が起こる範囲内ならば、精度良く成り立つと考えることができる。

\section{放射伝達方程式の導出}
ある方向に進む放射が、吸収や散乱などの相互作用で、どのように増減するか
記述するのが、放射伝達方程式である。

放射伝達方程式を導き、特定の状況での放射伝達方程式の解を求める。

\subsection{放射伝達方程式}
放射伝達強度の変化 $dI_\lambda$ を物質との相互作用による減少
$dI_\lambda^\mathrm{d}$ と、射出と多重散乱による増加
$dI_\lambda^\mathrm{s}$ に分けて考える
\begin{equation}
	dI_\lambda=dI_\lambda^\mathrm{d}+dI_\lambda^\mathrm{s}
\end{equation}

$dI_\lambda^\mathrm{d}$ について、
放射の伝搬方向に厚さ $ds$ で密度 $\rho$ の媒質を横切った後に、
放射強度が減少したとする
\begin{equation}
	dI_\lambda^\mathrm{d}=-k_\lambda\rho i_\lambda\,ds
\end{equation}
$k_\lambda$ は質量消散断面積と呼ばれる

以下の式によって放射源係数 $j_\lambda$ を定義する
\begin{equation}
	dI_\lambda^\mathrm{s}=j_\lambda\rho\,ds
\end{equation}

以上の式より
\begin{equation}
	dI_\lambda=dI_\lambda^\mathrm{d}+dI_\lambda^\mathrm{s}
	=-k_\lambda\rho I_\lambda\,ds+j_\lambda\rho\,ds
\end{equation}

放射源関数 $J_\lambda\equiv j_\lambda/k_\lambda$ を導入すると、
\begin{equation}
	\frac{dI_\lambda}{k_\lambda\rho\,ds}=-I_\lambda+J_\lambda\quad\text{……放射伝達方程式}
\end{equation}

\subsection{ビーアー・ブーゲー・ランバートの法則}
大気系からの射出の寄与を無視できる
\begin{equation}
	\frac{dI_\lambda}{k_\lambda\rho\,ds}=-I_\lambda
\end{equation}

積分すると
\begin{equation}
	I_\lambda[s_1]=I_\lambda[0]\exp[-k_\lambda u],\qquad u=\int^{s_1}_0\rho\,ds\quad\text{(光路長)}
\end{equation}

一様に消散する媒質を横切る放射強度の低下が、質量消散断面積と
光路長の積の指数関数に従う

\subsection{シュワルツシルトの方程式}
地球と大気から射出される熱赤外放射伝達\\
放射源関数はプランク関数で与えられる

放射伝達方程式: シュワルツシルトの式
\begin{equation}
	\frac{dI_\lambda}{k_\lambda\rho\,ds}=-I_\lambda+B_\lambda[T]
\end{equation}

シュワルツシルトの式の解:
\begin{equation}
	I_\lambda[s_1]=I_\lambda[0]\exp\bigl[-\tau_\lambda[s_1,0]\bigr]+
	\int^{s_1}_{0}B_\lambda\bigl[T[s]\bigr]\exp\bigl[-\tau_\lambda[s_1,s]\bigr]k_\lambda\rho\,ds
\end{equation}
\begin{equation}
	\tau_\lambda[s_1,s]=\int^{s_1}_s k_\lambda\rho\,ds\qquad\text{($s$ と $s_1$ の間の光学的厚さ)}
\end{equation}


\section{スペクトル線の広がり}
単色の放射は、相互作用により非単色となり、有限の幅を持つスペクトル線となる

\subsection{スペクトル線の広がり}
単色の射出は現実には観測されない

原子や分子への外部からの影響と、射出の際のエネルギー損失のため、
エネルギー遷移する間にエネルギー準位が僅かに変化する

その結果、非単色の放射となり、有限の幅を持つスペクトル線が観測される

\subsection{ローレンツ線形}
\begin{description}
	\item[ローレンツ線形] 衝突によって広げられたスペクトル線の線形
\end{description}

\begin{equation}
	k_\nu=\frac{S}{\pi}\frac{\alpha}{(\nu-\nu_0)^2+\alpha^2}=Sf[\nu-\nu_0]
\end{equation}
\centeralign{\includegraphics[width=\textwidth]{lorentz.jpg}}

$k_\nu$: 吸収係数 ;\quad
$\nu_0$: 理想的な単色の吸収線の波数;\\
$\alpha$: 吸収線の半値半幅(圧力と温度の関数);\\
$f$: 形状因子 (shape factor);\quad
$\displaystyle S=\int^\infty_{-\infty}k_\nu\,d\nu$: 線強度

\section{大気の熱赤外放射伝達}
放射伝達方程式から、大気の放射フラックス密度を形式的に得る


\subsection{放射伝達のための一般的な方程式}
放射束の放射強度: $I_\nu$;\quad 吸収係数: $k_\nu$;\\
吸収気体の密度: $\rho_a$;\quad 光路長: $s$;\quad 放射源関数: $J_\nu$
\begin{equation}
	-\frac{1}{k_\nu \rho_a}\frac{dI_\nu}{ds}=I_\nu-J_\nu
\end{equation}

\begin{description}
	\item[放射強度] 時間に依存しないと仮定
	\item[平行平面大気] 放射強度と大気パラメーターは鉛直方向にのみ変化
\end{description}

$\tau$ 座標に変換する
放射強度は天頂角と鉛直位置のみの関数\\
$B_\nu[z]=B_\nu[T[z]]$ はプランク放射強度
\begin{equation}
	-\mu\frac{dI_\nu[z,\mu]}{k_\nu\rho_a\,dz}=I_\nu[z,\mu]-B_\nu[z]
\end{equation}

光学的深さ $\tau$ を導入($p=\rho_a/\rho$ は気体の混合比)
\begin{equation}
	\tau=\int^{\mathrm{TOA}}_{z} k_\nu[z]\rho_a[z]\,dz=\int^p_0 k_\nu[p]q[p]\frac{dp}{g}
\end{equation}
\begin{equation}
	d\tau=-k_\nu[z]\rho_a[z]\,dz=k_\nu[p]q[p]dp/g
\end{equation}

放射伝達方程式を $\tau$ 座標に変換
\begin{equation}
	\mu\frac{dI_\nu[\tau,\mu]}{d\tau}=I_\nu[\tau,\mu]-B_\nu[\tau]
\end{equation}

\subsection{境界条件}
境界条件: 地球表面からの射出と大気上端 (TOA) からの射出が等方性となる

全光学的厚さ: $\tau_*$

地球表面は赤外で黒体と仮定 $I_\nu[\tau_*,\mu]=B_\nu[\tau_*]$

大気上端では $I_\nu[0,-\mu]=B[\mathrm{TOA}]\simeq0$ と仮定

\subsection{放射強度の型式解}
単色の透過率 $T_\nu[\tau/\mu]=e^{-\tau/\mu}$ を定義

放射強度の型式解は
\begin{align}
	I^\uparrow_\nu[\tau,\mu]
		&=B_\nu[\tau_*]T_\nu\left[\frac{\tau_\nu-\tau}{\mu}\right]
		-\int^{\tau_*}_\tau B_\nu[\tau']\frac{d}{d\tau'}T_\nu\left[\frac{\tau'-\tau}{\mu}\right]d\tau'\\
	I^\downarrow_\nu[\tau,-\mu]
		&=\int^\tau_0 B_\nu[\tau']\frac{d}{d\tau'}T_\nu\left[\frac{\tau-\tau'}{\mu}\right]d\tau'
\end{align}

大気の加熱率の計算に必要な放射フラックス密度は、上半球と下半球の放射強度の和
\begin{equation}
	F^{\uparrow\downarrow}_\nu[\tau]=2\pi\int^1_0 I^{\uparrow\downarrow}_\nu[\tau,\pm\mu]\mu\,d\mu
\end{equation}

\subsection{大気の放射フラックス密度}
角度方向の積分を考慮するため、拡散透過率 $T^f_\nu$ を導入
\begin{equation}
	T^f_\nu[\tau]=2\int^1_0 T_\nu\left[\frac{\tau}{\mu}\right]\mu\,d\mu
\end{equation}

\begin{align}
	F^\uparrow_\nu[\tau]
		&=\pi B_\nu[\tau_*]T^f_\nu[\tau_*-\tau]
		-\int^{\tau_*}_\tau \pi B_\nu[\tau']\frac{d}{d\tau'}T^f_\nu[\tau'-\tau]d\tau'\\
	F^\downarrow_\nu[\tau]
		&=\int^\tau_0 \pi B_\nu[\tau']\frac{d}{d\tau'}T^f_\nu[\tau-\tau']d\tau'
\end{align}

放射フラックスの波数積分は、$\tau$ が波数の関数なので、
\begin{equation}
	F^{\uparrow\downarrow}[z]=\int^\infty_0 F^{\uparrow\downarrow}_\nu[z]\,d\nu
\end{equation}
光学的深さに沿った波数積分を必ず含む

したがって、吸収係数は吸収線強度と吸収線形の式で表すことができる
\begin{equation}
	k_\nu[p,T]=\sum_j S_j[T]f_{\nu,j}[p,T]
\end{equation}

\subsection{分光透過率}
赤外放射伝達の計算をする際は、プランク関数の変動が無視しうるような
小さな波数区間で放射パラメーターを決めることが有効

放射強度と放射フラックスの方程式における基本的な放射パラメーターとして、
平均波数の添字 $\bar\nu$ のついた分光透過率を定義する
\begin{equation}
	T_{\bar\nu}[u]
	=\int_{\Delta\nu}\exp[-\tau]\frac{d\nu}{\Delta\nu}
	=\int_{\Delta\nu}\exp\left[-\int_u\sum_j k_{\nu,j}[u]\,du\right]\frac{d\nu}{\Delta\nu}
\end{equation}

\section{相関 $k$ 分布法}

気体の分光透過率を吸収係数 $k_\nu$ に応じてグループとして扱う

均質な大気では分光透過率は吸収係数 $k$ の順序に依存しない\\
波数積分は $k$ 空間での積分に置き換えることができる

波数区間 $\Delta\nu$ における $k_\nu$ に対する正規化確率密度関数が
$f[k]$ で与えられ、その最大値と最小値がそれぞれ $k_{\mathrm{min}}\to0$、
$k_{\mathrm{max}}\to\infty$ だとすると、分光透過率は
\begin{equation}
	T_{\bar\nu}[u]=\int_{\Delta\nu}\exp[-k_\nu u]\frac{d\nu}{\Delta\nu}=\int^\infty_0 \exp[-ku]f[k]\,dk
\end{equation}

確率密度関数 $f$ は分光透過率のラプラス逆変換であるとわかる
\begin{equation}
	f[k]=\mathcal{L}^{-1}[T_{\bar\nu}[u]]
\end{equation}

累積確率分布関数を定義する
\begin{equation}
	g[k]=\int^k_0 f[k]\,dk
\end{equation}

$g[0]=0,\ g[k\to\infty]=0,\ dg=f\,dk$ より、分光透過率は次で表すことができる
\begin{equation}
	T_{\bar\nu}[u]=\int^1_0 \exp[-k[g]u]\,dg\simeq\sum_j\exp[-k[g_j]u]\,\Delta g_j
\end{equation}

\subsection{不均質大気への応用}
相関 $k$ 分布法は $\nu$ 積分を $g$ 積分で置き換える\\
吸収係数が一定であると仮定している

現実大気では、吸収係数は圧力に応じて変化するので、
鉛直方向の吸収係数の変化を考えなければならない
\begin{equation}
	T_{\bar\nu}[u]
	=\int_{\Delta\nu}\exp\left[-\int_u k_\nu\,du\right]\frac{d\nu}{\Delta\nu}
	\stackrel{?}{=}\int^1_0\exp\left[-\int_uk[g]\,du\right]dg
\end{equation}

波数区間 $-\Delta\nu/2$ から $\Delta\nu/2$ で単一の吸収線を考える

吸収線の中心では $\nu[k_{\mathrm{max}}]=0$、吸収線の端では
$|\nu[k_{\mathrm{min}}]|=\Delta\nu/2$ となり、累積密度関数は以下になる
\begin{equation}
	g[k]=\int^k_0 f[k]\,dk
	=\frac{2}{\Delta\nu}\int^k_{k_{\mathrm{min}}}\left|\frac{d\nu}{dk}\right|_{k=k'}dk'
	=\frac{2}{\Delta\nu}\nu[k]-1
\end{equation}

すなわち、単一の吸収線の場合は $\nu$ 積分を $g$ 積分に置き換えることができる

波数区間 $\Delta\nu$ 内で周期的に生起される吸収線を考える

吸収線の間隔が $\delta$ であるとすると、
\begin{equation}
	g[k]=\int^k_0 f[k]\,dk
	=\frac{2}{\delta}\int^k_{k_{\mathrm{min}}}\left|\frac{d\nu}{dk}\right|_{k=k'}dk'
	=\frac{2}{\delta}\nu[k]-1
\end{equation}

したがって、この場合も $\nu$ 積分を $g$ 積分に置き換えることができる

\section{まとめ}

\begin{itemize}
	\item 基礎方程式の概念がわかった
	\item 特定の状況(大気系からの射出を無視できる場合や、シュワルツシルトの方程式)
		での、放射伝達方程式の解法を学んだ
	\item 放射伝達方程式から計算できるもの
		\begin{itemize}
			\item 放射伝達方程式から吸収の鉛直分布を求めることができる
			\item その結果、気温の鉛直分布がわかるようになる
		\end{itemize}
	\item 実際に応用する方法がわかっていないので、第 8 章を読み進めているところ
\end{itemize}

\begin{itemize}
	\item 大気放射フラックスを計算する方法について学んでいる
	\item 相関 $k$ 分布法は $\nu$ 積分を $g$ 積分に置換し、
		計算量を少なくすることができる
\end{itemize}

\begin{itemize}
	\item 相関 $k$ 分布法で計算をする手順を学んでいるので、
		実際にそれを用いた計算を行いたい
\end{itemize}

\begin{thebibliography}{9}
	\bibitem{liou} 大気放射学 ---衛星リモートセンシングと気候問題へのアプローチ---
		K. N. Liou(著) 藤枝 鋼、深堀 正志(翻訳)
		共立出版 \textbf{(2014)} 672ページ
\end{thebibliography}

\end{document}
